\documentclass[
    linespread = 1.25
]{ctexart}
\pagestyle{plain}
\ctexset{
    section/format = \Large\bfseries\raggedright,
    section/number = {\chinese{section}、},
    section/aftername = {\enskip},
    abstractname = {\zihao{-2}摘\quad 要}
}

\usepackage[a4paper, lmargin=1in, rmargin=1in, tmargin=1in, bmargin=1in]{geometry}
\usepackage{amsmath}
\usepackage{booktabs}
\usepackage{graphicx}
\graphicspath{ {./fig} }

\usepackage{listings}
\usepackage{color}

\definecolor{dkgreen}{rgb}{0,0.6,0}
\definecolor{gray}{rgb}{0.5,0.5,0.5}
\definecolor{mauve}{rgb}{0.58,0,0.82}

\lstset{
  frame=tb,
  language=Python,
  aboveskip=3mm,
  belowskip=3mm,
  showstringspaces=false,
  columns=flexible,
  basicstyle={\small\ttfamily},
  numbers=left,
  numberstyle=\tiny\color{gray},
  keywordstyle=\color{blue},
  commentstyle=\color{dkgreen},
  stringstyle=\color{mauve},
  breaklines=true,
  breakatwhitespace=true,
  tabsize=2
}

\usepackage[hidelinks]{hyperref}
\usepackage{caption}
\usepackage{subcaption}
\usepackage{siunitx}
\usepackage{algorithm2e}
\SetAlgoInsideSkip{bigskip}
\SetAlgorithmName{算法}{算法}{算法}
\RestyleAlgo{ruled}
\usepackage{multicol}
\usepackage{longtable}
\usepackage{tablefootnote}

\title{\zihao{2}\textbf{大数据创新实践实验报告}\\\zihao{3}\textbf{——多模态大模型微调}}
\author{\zihao{4}曹瀚文 \\\texttt{学号:210810503}
\and \zihao{4}岑畅 \\\texttt{学号:210810501}
\and \zihao{4}丁有罡 \\\texttt{学号:210810518}
\and \zihao{4}符永宣\\\texttt{学号:210810506}
\and \zihao{4}金文韬\\\texttt{学号:210810306}
\and \zihao{4}刘炎培\\\texttt{学号:210810510}
\and \zihao{4}刘梓涛\\\texttt{学号:210810513}
\and \zihao{4}彭珂\\\texttt{学号:210810508}
\and \zihao{4}王子霖\\\texttt{学号:210810522}
\and \zihao{4}文宇祥\\\texttt{学号:210810514}
}
\date{}

\begin{document}

\begin{titlepage}
\newgeometry{top=1in,bottom=1in,right=0.75in,left=0.75in}
\maketitle
\vspace{0.2cm}
\begin{abstract}
  \zihao{-4}
  \vspace{0.8cm}
  \linespread{1.25}
  本论文主要研究了空气质量、污染物水平及其与时空、气候因素的关系,并基于历史数据预测未来空气质量。论文首先对数据进行了预处理,包括数据描述、数据标准化、异常值及缺失值处理、极值值处理等步骤。接着,采用系统聚类方法对不同城市的污染物水平进行了潜在模式探索,通过均值和不同邻距离的聚类方法分析得出了系统聚类结果,并用因子分析的得分对其进行了解释。

  论文进一步探讨了空气质量与时空、气候因素的相关关系,应用多元线性回归模型和改进的多元线性回归模型,诊断模型结果并分析了气候因素与空气质量之间的相关性。此外,论文介绍并应用了广义线性模型,实现利用时空、气候因素对空气质量进行更加准确的预测。
  
  最后,论文利用SARIMA模型和GARCH模型,根据历史空气质量数据预测未来一段时间内的空气质量,以南昌市为例进行了实证研究。通过数据导入、探索性期性、参数确定、模型诊断及预测等步骤,详细展示了两种模型的应用过程和预测效果。
  
  本文不仅揭示了不同城市空气污染物可能存在的潜在模式,同时探究了空气质量与多种因素之间的复杂关系,也为空气质量的预测提供了有效的模型和方法,对城市环境管理、污染控制和空气质量的预报具有重要的参考价值。

  \vspace{1cm}
  \noindent\textbf{关键词:} 系统聚类\hspace{0.22cm} 因子分析\hspace{0.22cm} 多元线性回归\hspace{0.22cm} 广义线性模型\hspace{0.22cm} 时间序列分析
\end{abstract}
\end{titlepage}

\tableofcontents
\newpage
\section{实验背景}

\section{基于自动驾驶数据集的模型微调}

\section{探索影响微调效果的实验因素}

\section{微调结果的初步评估}

\section{实验结论}

\appendix
\newpage
\section*{参考文献}
\addcontentsline{toc}{section}{参考文献}
\noindent
[1] 乐东明,王文浚,王颖,等. 2020—2022年咸宁市臭氧污染气象特征及成因分析[J].黑龙江环境通报, 2024, 37(05):30-32.

\noindent
[2] 李高荣,吴密霞. 多元统计分析[M]. 北京:科学出版社, 2021

\noindent
[3] John A. Rice. Mathematical Statistics and Data Analysis[M]. Boston: Cengage Learning, 2006

\noindent
[4] 何书元. 应用时间序列分析[M]. 北京:北京大学出版社, 2003

\newpage
\section*{附录:实验日志与心得}
\addcontentsline{toc}{section}{附录:实验日志与心得}

\end{document}

